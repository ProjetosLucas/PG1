% Ajustar esse \vspace de acordo com o necessário
\vspace{-42pt}
\section[Objetivos Geral]{Objetivos Geral}
O objetivo geral deste trabalho é o desenvolvimento de um sistema de controle para equipamentos dos Laboratórios da Engenharia Elétrica da Ufes. Este sistema obtém, a partir de informações de dados, a data do produto (patrimonial) e o estudante (RA), informações adicionais adicionais de continuação de propriedades.

O projeto em desenvolvimento para a disciplina de Projeto Orientado tem como objetivo a resolução de alguma problemática cotidiana por vias de Internet das coisas (do inglês, Internet of Things - IoT)[1][3]. Tomando como base a ideia acordada pelo professor e alunos da disciplina, o grupo propôs desenvolver um projeto que apresente melhorias em economia de tempo e trabalho humano dos laboratórios de eletrônica do prédio do CT II. 

Tais laboratórios são constantemente utilizados e, dessa forma, o grupo se propôs a pensar em uma solução e executá-la de forma que auxilie o processo de gerenciamento de KITs dos laboratórios, ajudando tanto alunos e professores como os próprios funcionários do local. A ideia tem aplicabilidade em diversas instâncias, para tanto, tal ideia foi generalizada para gerenciamento de KITs, de forma a atender outras áreas e não somente os laboratórios do CT II. 

Atualmente os Kits ficam armazenados no almoxarifado do laboratório e quando são emprestados é preciso que o aluno entregue algum documento com foto (Carteira de Identidade ou Carteira Estudantil) onde o funcionário do laboratório guarda o documento junto com uma ficha, sendo entregue somente quando o estudante devolver o KIT. Isso gera confusão para ser pegar e entregar dos documentos e demora com filas de estudantes. Por isso é necessário que seja feito um controle mais aprimorado desses empréstimos.Além disso, o sistema manual não gurada o histórico de empréstimo dos equipamentos. 

O projeto proposto tem como objetivo geral a simplificação e automatização do gerenciamento de empréstimos de livros e equipamentos em bibliotecas e empresas. Como objetivos específicos temos o intuito de registrar toda uma coleção de KITs didáticos para aulas de eletrônica nos laboratórios do CT II e ter controle com um cadastro de usuários.
Um sistema automatizado se ganha agilidade, maior segurança nos dados do empréstimo e ainda mantém um histórico sempre atualizado dos empréstimos de cada KIT.


\section[Objetivos Específicos]{Objetivos Especificos}
Os objetivos específicos desse projeto foram divididos em alguns tópicos, os quais estão listados nos tópicos a seguir:

\begin{itemize}
   \item Desenvolver um software na linguagem C para fazer a leitura dos códigos de barra do RA e do patrimônio do equipamento por meio de um leitor conectado a Raspberry Pi;
   \item Modelar um sistema de banco de dados, o qual irá armazenar os dados dos empréstimos;
   \item Desenvolver uma interface web para fazer o controle dos empréstimos de maneira automatizada. Através dessa interface, o usuário poderá controlar todo o sistema e terá acesso a todos os relatórios desejados;
   \item Desenvolver uma forma de integração entre a interface web e o software que faz as
leituras dos códigos de barra;
   \item Fazer as verificações necessárias no sistema e por fim validar o seu funcionamento;
   \item Com o sistema funcionando, desenvolver um script para a sua instalação;
   \item Desenvolver um manual de operação do sistema para fornecer para o usuário.
\end{itemize}
