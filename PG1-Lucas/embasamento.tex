% Ajustar esse \vspace de acordo com o necessário
\vspace{-42pt}
\section[Automatização de processos]{Automatização de processos}
Automatizar um processo consiste em defini-los e otimizá-los para em seguida executálos sobre uma plataforma informatizada (CAPIOTTI, 2012). 

Para a automação é necessárioinicialmente elaborar a chamada arquitetura de processos. Essa arquitetura nada mais é do quea estrutura geral de um sistema de processos combinada com o conceito de automatizaçãoaplicável a ela (DAWIS, 2001).No meio industrial, a preocupação com produtividade, redução do risco operacional equalidade, leva à implantação de sistemas de automatização. 

Esses sistemas visam melhorar osprocessos industriais e também auxiliam na identificação de indicadores de desempenho doprocesso, o que permite aperfeiçoamento constante das atividades dos processos(SGANDERLA, 2013).

\section[Aplicação WEB]{Aplicação WEB}
Uma aplicação web pode ser definida como uma aplicação que é acessada através do navegador (NATIONS, 2016). Ela executa tarefas em um servidor e faz a interface com o usuário através de uma página web (PALMEIRA, 2013). Essas aplicações são desenvolvidas com o auxílio de diversas tecnologias, como linguagens de programação (PHP, Javascript, etc), elementos de interface gráfica (HTML, CSS) e também técnicas como o AJAX (Asynchronous JavaScript And XML).

As aplicações web se diferenciam das aplicações ‘desktop’ pois não é necessário que se desenvolva um programa específico para cada hardware ou sistema operacional, basta acessálas via navegador web. Isso é possível pois todo o processamento de funções e instruções é feito no servidor web e o navegador funciona apenas como uma ‘interface’ da  plicação (NATIONS, 2016). O funcionamento de uma aplicação web pode ser melhor compreendido na Figura 2, nela é ossível observar um exemplo básico das etapas de execução de uma aplicação Web.