% Ajustar esse \vspace de acordo com o necessário
\vspace{-42pt}
\section[Automatização de processos]{Automatização de processos}
Automatizar processos significa passar as tarefas realizadas de maneira manual pelas pessoas para equipamentos, máquinas, instrumentos e outros \cite{AutomatProcess}. Para que a automatização de processos ofereça os resultados esperados, é muito importante garantir que sua implantação seja feita de maneira estruturada e de acordo com as diretrizes de onde está sendo aplicado \cite{AutomatProcess2}.No meio industrial, a preocupação com produtividade, redução do risco operacional e qualidade, leva à implantação de sistemas de automatização. 

A parte operacional na automação industrial é uma parte do sistema que atua diretamente no processo e é um conjunto de elementos que fazem com que a máquina se mova e realize a operação desejada \cite{AutomatProcess3}, aperfeiçoamento constante das atividades dos processos.

Automação em processos indutrias foram abordadas nas disciplinas de Controle Inteligente e Sistemas Realimentados, onde eram abordados diversos meios de controlar o processo. Neste projeto desenvolverá principalmente um sistema de software e hardware para automatizar o processo de emprétimos de equipamentos do laboratório.

\section[Aplicação WEB]{Aplicação WEB}
uma aplicação web é um software que é instalado em um servidor web e é projetado para responder a solicitações, processar informações, armazenar informações e dimensionar as respostas de acordo com a demanda e, em muitos casos, é distribuído em vários sistemas ou servidores \cite{WEB1}. Essas aplicações apresenta várias linguagens de programação (PHP, Javascript, etc) e elementos de interface gráfica (HTML, CSS).

As aplicações web se diferenciam das aplicações ‘desktop’ pois não precisa de instalação no computador, acessíveis de qualquer lugar com Internet, não depende de sistema operacional tendo todo o processamento de funções e instruções feito no servidor web e o navegador funciona apenas como uma ‘interface’ da  aplicação \cite{WEB2}. Essas vantagens de aplicação WEB foram vistas principalmente na displina  Redes de Computadores e de Automação.

Há a existencia de frameworks. Um framework em desenvolvimento de software, é uma abstração que une códigos comuns entre vários projetos de software provendo uma funcionalidade genérica \cite{WEB3}. Assim teremos para o desenvolvimento para esse software a framework chamada CakePHP \cite{Cake} que  torna a construção de aplicativos da web mais simples, mais rápida e requer menos código. 

\section[Banco de Dados]{Banco de Dados}
Um banco de dados é uma coleção de dados inter-relacionados, representando informações sobre um domínio específico, ou seja, sempre que for possível agrupar informações que se relacionam e tratam de um mesmo assunto, posso dizer que tenho um banco de dados \cite{BD1}. Como exemplo de banco de dados podemos citar um sistema de bibliotecas, uma agenda telefônica, um cadastro de clientes, etc.

Um sistema de gerenciamento de banco de dados (SGBD) é um software que possui recursos capazes de manipular as informações do banco de dados e interagir com o usuário. Existem vários SGBD’s no mercado, como Oracle, SQL Server, DB2, PostgreSQL, MySQL, o próprio Access ou Paradox, entre outros \cite{BD1}.

Os sistemas de banco de dados têm certas vantagens em relação ao sistema tradicional de armazenamento de arquivos. Eles são implementados com a função de isolar os detalhes internos do banco de dados do usuário, ou seja, promover a abstração desses dados e também permitir a relativa dependência dos dados e aplicativos que acessam \cite{BD1}.

Outro fator importante é a questão da segurança e integridade dos dados, pois estes são geralmente criptografados e não são acessados tão facilmente. No entanto, a implantação de um sistema de banco de dados é mais cara e nem sempre é necessário usá-lo \cite{BD1}.

Para realizar consultas, inserir, editar e vincular dados armazenados no banco de dados, é usada uma linguagem baseada em consultas estruturadas chamada SQL (Structured Query Language) \cite{BD1}.

A importancia em banco de dados foi abordado principalmente em disciplinas como Controle Inteligente. O banco de dados será  utilizado para armazenar os dados do empréstimos de equipamentos. SGBD utilizado será o MySQL, devido o fato de estar presento no XAMPP, que é um pacote com os principais servidores de código aberto do mercado, utilizado para o desenvolvimento da interface WEB \cite{BD2}.

\section[Sistema Embarcado]{Sistema Embarcado}

Um sistema embarcado consiste em uma união de hardware, software e em alguns casos, elementos mecânicos, que são utilizados para realizar uma determinada tarefa (STADZISZ; RENAUX, 2008). Com o advento da eletrônica e a redução dos custos, a utilização dos sistemas embarcados aumentou significativamente nos últimos anos. Para se ter uma noção, 99% de toda a produção mundial de microprocessadores é utilizada em sistemas embarcados e apenas 1% é destinada a computadores (BARR; MASSA, 1999). 

Sistemas embarcados podem ser utilizados em diversas aplicações. Isso é possível devido à enorme variedade de processadores disponíveis no mercado, o que leva ao desenvolvimento de vários sistemas, os quais vão desde uma simples escova de dentes eletrônica, até o painel de controle de um avião. Apesar dessa variedade, o desenvolvimento de sistemas embarcados pode se tornar complicado devido às restrições de cada um. 

Essas restrições, de acordo com Stadzisz e Renaux (2008) implicam numa grande diferença entre os sistemas embarcados e os computadores convencionais. entre elas pode-se citar as chamadas restrições dimensionais, as quais envolvem tamanho e peso, sendo de extrema importância em equipamentos de pequeno porte como um celular. Uma outra restrição é a de consumo de energia, a qual é de extrema importância em equipamentos móveis e que são alimentados por meio de baterias, como o caso de um aparelho de GPS. As restrições de recursos, como memória e processamento geram um impacto na construção do software, ou seja, é preciso ter um software eficiente para que o funcionamento do sistema não apresente problemas. Uma outra restrição que pode ser citada é a de tempo de execução. Ela é pertinente pois existem diversas aplicações que precisam ser executadas em um limite de tempo bem específico.

O projeto de um sistema embarcado é extremamente ligado à sua aplicação, portanto não é possível descrever uma rotina genérica para o desenvolvimento de um projeto. Uma opção muito usada por desenvolvedores é a utilização de placas de desenvolvimento, como a Raspberry Pi, a Beaglebone, o Arduino, etc. Essas placas possuem vários periféricos e são utilizadas para facilitar e agilizar o processo de prototipação. Posteriormente é possível otimizar
os recursos e dispensar os periféricos desnecessários e projetar um hardware específico para a aplicação desejada.