% Ajustar esse \vspace de acordo com o necessário
\vspace{-42pt}
\section[Automatização de processos]{Automatização de processos}
Automatizar um processo consiste em defini-los e otimizá-los para em seguida executálos sobre uma plataforma informatizada (CAPIOTTI, 2012). 

Para a automação é necessárioinicialmente elaborar a chamada arquitetura de processos. Essa arquitetura nada mais é do quea estrutura geral de um sistema de processos combinada com o conceito de automatizaçãoaplicável a ela (DAWIS, 2001).No meio industrial, a preocupação com produtividade, redução do risco operacional equalidade, leva à implantação de sistemas de automatização. 

Esses sistemas visam melhorar osprocessos industriais e também auxiliam na identificação de indicadores de desempenho doprocesso, o que permite aperfeiçoamento constante das atividades dos processos(SGANDERLA, 2013).

\section[Aplicação WEB]{Aplicação WEB}
Uma aplicação web pode ser definida como uma aplicação que é acessada através do navegador (NATIONS, 2016). Ela executa tarefas em um servidor e faz a interface com o usuário através de uma página web (PALMEIRA, 2013). Essas aplicações são desenvolvidas com o auxílio de diversas tecnologias, como linguagens de programação (PHP, Javascript, etc), elementos de interface gráfica (HTML, CSS) e também técnicas como o AJAX (Asynchronous JavaScript And XML).

As aplicações web se diferenciam das aplicações ‘desktop’ pois não é necessário que se desenvolva um programa específico para cada hardware ou sistema operacional, basta acessálas via navegador web. Isso é possível pois todo o processamento de funções e instruções é feito no servidor web e o navegador funciona apenas como uma ‘interface’ da  plicação (NATIONS, 2016). O funcionamento de uma aplicação web pode ser melhor compreendido na Figura 2, nela é ossível observar um exemplo básico das etapas de execução de uma aplicação Web.

\section[Banco de Dados]{Banco de Dados}
Um banco de dados pode ser definido como um conjunto de dados inter-relacionados que representam informações sobre um domínio específico, ou seja, sempre que for possível agrupar informações que se relacionam, existe um banco de dados (KORTH, 1994). Como exemplo de banco de dados podemos citar um sistema de bibliotecas, uma agenda telefônica,
um cadastro de clientes, etc.

Para fazer o gerenciamento do banco de dados, são utilizados os sistemas de gerenciamento de banco de dados (SGBD). Esses sistemas nada mais são do que um software que é capaz de manipular as informações do banco de dados e fazer a interação com o usuário.

Existem vários SGBD’s no mercado, como o MySQL, o SQL Server, o PostgreSQL, entre outros (REZENDE, 2006).
Um sistema de banco de dados pode ser definido como um conjunto de quatro componentes básicos: os dados, o hardware, o software e os usuários. Para compreender melhor esses componentes, apresenta-se o diagrama da Figura 5.

Os sistemas de banco de dados possuem certas vantagens em relação ao sistema tradicional de armazenamento de arquivos. Eles são implementados com a função de isolar os detalhes internos do banco de dados do usuário, ou seja, promover a abstração desses dados e também de permitir que exista uma independência dos dados em relação às aplicações que os
acessam. 

Outro fator importante é a questão da segurança e integridade dos dados, pois geralmente eles são criptografados e não podem ser acessados tão facilmente. No entanto, a implantação de um sistema de banco de dados é mais cara e nem sempre é necessário que ele seja utilizado.

Para realizar consultas, inserção, edição e relacionar os dados armazenados no banco de dados, utiliza-se uma linguagem baseada em consultas estruturadas chamada SQL (Structured Query Language).