% Ajustar esse \vspace de acordo com o necessário
\vspace{-42pt}
Os equipamentos do Laboratório do prédio da Engenharia Elétrica são de alto custo e são utilizados nas aula práticas, o que torna necessário ter um controle minimamente estruturado dos empréstimos para que não tenha percas, prejuízos e aulas práticas com alunos sem equipamentos. Para ter este controle, os dados do equipamento e do aluno que solicitou devem ser armazenado corretamente para que, caso ocorra algum problema com o equipamento, sejam tomadas atitudes necessárias para resolvê-lo, ou caso seja utilizado em alguma outra aula, não possa ser emprestado. O controle manual desses empréstimos é efetivo até certo ponto, mas pode apresentar alguns erros.

Com um sistema automatizado ganha agilidade, maior segurança na data do empréstimo e ainda manter um histórico atualizado de cada um dos equipamentos de empréstimos. Uma implementação de um sistema eletrônico para controlar os empréstimos é necessário para que os dados sejam registrados corretamente.

Manter um histórico de empréstimos é importante para evitar o uso excessivo do mesmo equipamento, ou seja, nem sempre emprestar equipamentos mesmo para evitar o seu desgaste excesivo. Já existem sistemas que possam ser utilizados, mas muitos são desenvolvidos para bibliotecas, possuindo funções excedentes e alto custo. Desta forma, o sistema será desenvolvido para automatizar o processamento de empréstimos do equipamento e armazenar o histórico. 

Também podem ser feitos diversas funções em um sistema automatizado de BD (Banco de Dados) que facilitam na visualização dos dados de emprestimos: listar os equipamentos utilizados por aula, por aluno, ou sala; diversas formas de gerar relatório dos itens emprestados; e histórico de empréstimo por estudante.

Também esse sistema irá permitir que equipamentos sejam utilizados fora do laboratório, pois poderá monitorá a suas localizações através de um sistema com GPS que alimenta o banco de dados com a posições geográficas dos equipamentos.

Deve-se pesquisar a forma que será feita o acesso a interface WEB e ao banco de dados, quais são as pessoas que podem utilizá-las e como irão usá-las. É presciso encontrar também uma melhor forma de projetar um sistema de GPS que seja barato e discreto para serem colocados em todos os equipamentos do laboratório.

Uma limitação desse projeto é pricipalemente o custo dos circuitos implementados em todos equipamentos e do sistema web fosse acessada remotamente em todas as parte do mundo. Outra limitação também é o tempo para seja desenvolvido, o que pode não conseguir atingir todos os requisitos iniciais do projeto.