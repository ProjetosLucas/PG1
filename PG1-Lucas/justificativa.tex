% Ajustar esse \vspace de acordo com o necessário
\vspace{-42pt}
Como a DAELN possui equipamentos de alto custo, é necessário ter um controle minimamente estruturado dos empréstimos. Para ter este controle, os dados do equipamento e o aluno que solicitou ser armazenado corretamente para que, caso ocorra algum problema com o equipamento, sejam tomadas atitudes necessárias para resolvê-lo. O controle manual desses empréstimos é efetivo até certo ponto, mas pode apresentar alguns erros.

Com um sistema automatizado para ganhar agilidade, maior segurança na data do empréstimo e ainda manter um histórico atualizado de cada um dos equipamentos de empréstimos. Uma implementação de um sistema eletrônico para controlar os empréstimos necessários para que os dados sejam registrados corretamente, e isso acabará evitando tempestades no futuro.

Manter um histórico de empréstimos é importante para evitar o uso excessivo do mesmo equipamento, ou seja, nem sempre emprestar equipamentos mesmo para evitar o seu desgaste excesivo.Como uso já foi dito antes, até que existam sistemas que possam ser utilizados, mas eles não são desenvolvidos para isto, possuindo funções excedentes e alto custo. Desta forma, o sistema foi desenvolvido para automatizar o processamento de empréstimos do equipamento e armazenar o histórico do mesmo.