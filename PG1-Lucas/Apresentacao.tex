% Ajustar esse \vspace de acordo com o necessário
\vspace{-42pt}
Teste de citação: \cite{DepEngEle}

O Departamento de Engenharia Elétrica da Ufes possui diversos equipamentos, que podem ser utilizados durante as aulas práticas e também podem ser emprestados aos alunos em períodos fora do horário de aula para projetos a serem desenvolvidos, conclusão do curso (TCC). , projetos integrativos (PI), etc. Estes equipamentos costumam ter um alto custo, portanto, devemos ter muito cuidado com eles. O uso do equipamento é de extrema importância no desenvolvimento acadêmico dos alunos.
	
Segundo Varela (2009), a educação não é apenas sobre a teoria apresentada em sala de aula, ela também está diretamente relacionada aos recursos que são fornecidos para que os alunos possam desenvolver seu potencial. O uso de equipamentos não deve se restringir a aulas práticas com professores, mas é necessário ter controle sobre o empréstimo destes equipamentos. O equipamento disponível é armazenado na loja de departamento e pode ser emprestado aos alunos através da apresentação do registro acadêmico (RA).

Esse processo de empréstimo de equipamentos é controlado apenas por um token e isso acaba gerando confusão nas notas, sendo necessário, portanto, um melhor controle desses empréstimos. O controle manual consiste em anotar o RA do aluno e o patrimônio do equipamento, assim como a data da retirada e depois a data do retorno. No entanto, o sistema, por vezes, acaba por ser uma brecha para alguns problemas, tais como anotação errônea de dados, não-anotação de um retorno, entre outros.

Além disso, o sistema manual não leva o histórico de equipamentos e o mesmo acesso, o acesso aos documentos não é mais tão fácil quanto um sistema automatizado. do produto e datas no banco de dados, que foi oferecido em todos os momentos pelo manual de operação.

Além disso, o sistema guardião é um histórico de todos os empréstimos feitos para o acesso necessário. A automação do empréstimo pode ser feita com um sistema existente, como Aleph, Scobi, Pergamum, entre outros, mas esses sistemas são desenvolvidos para o armazenamento de bibliotecas, o que faz com que tenham funções que não são o caso do controle de empréstimos de equipamentos. . Além disso, os sistemas costumam ter um alto custo de implementação e alguns casos de hardware, o que aumenta ainda mais o custo (RODRIGUES e PRUDENCIO, 2009).

Trabalho atual com o desenvolvimento de um sistema de baixo custo para automatizar o processo de empréstimo associado ao dos Laboratórios da Engenharia Elétrica da Ufes.