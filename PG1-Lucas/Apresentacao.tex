% Ajustar esse \vspace de acordo com o necessário
\vspace{-42pt}
% Teste de citação: \cite{DepEngEle}
% Teste de citação: \cite{SistEmprestimo}
% Teste de citação: \cite{Laravel}
% Teste de citação: \cite{Cake}
% Teste de citação: \cite{Software}
% Teste de citação: \cite{LandItens}
% Teste de citação: \cite{SistBib}

O Departamento de Engenharia Elétrica da Ufes possui diversos equipamentos que podem ser utilizados nos laboratórios durante as aulas práticas ou fora aula  para projetos a serem desenvolvidos como trabalho de conclusão do curso (TCC), trabalhos de diversas disciplinas, etc. Estes equipamentos costumam ter um alto custo, portanto, devemos ter muito cuidado com eles. O uso do equipamento é de extrema importância no desenvolvimento acadêmico dos alunos.
	

Durante as aulas práticas acompanhada por professores no laboratório, há empréstimos de Kits com cabos de osciloscópio e de fonte de alimentação e protoboard. Esse processo de empréstimo de Kits é controlado manualmente. Para adquirir o Kit, o aluno tem que entregar um documento com foto e só será devolvido na devolução do Kit. O funcionário fixa no documento do aluno uma ficha com número referente ao Kit emprestado. Esse processo acaba gerando filas no laboratório e confusão na organização dos documentos dos alunos. 

Ainda não há um controle para equimentos que serão utilizados fora dos laboratórios. Deve ser montando um sistema que auxilie e controle esses empréstimos.  O controle manual normalmente se faz anotando o número de identificação do aluno, número de identificação do equipamento, a data de retirada e posteriormente a data de devolulução. Com esse sistema pode haver problemas com anotações errada dos dados, não dá para gerar e analisar o histórico de equipamentos e também não tem como saber a localização que ele se encontra. 

É presciso fazer um sistema que seja capaz de registrar os equipamentos de forma informatizada e que seja possível saber a sua localização, tirando todos os problemas causados pelo sistema manual. É presciso automatizar.

A automação do empréstimo pode ser feita com um sistema existente, como Aleph, Scobi, Pergamum, entre outros, porém os sistemas costumam ter um alto custo de implementação e alguns casos de hardware, o que aumenta ainda mais o custo.

Trabalho atual tem como ideia  desenvolver de um sistema de baixo custo para automatizar o processo de empréstimo de equipamentos e  Kits presentes no laboratórios da Engenharia Elétrica da Ufes.