% Ajustar esse \vspace de acordo com o necessário
\vspace{-42pt}
Teste de citação: \cite{DepEngEle}
Teste de citação: \cite{SistEmprestimo}
Teste de citação: \cite{Laravel}
Teste de citação: \cite{Cake}
Teste de citação: \cite{Software}
Teste de citação: \cite{LandItens}
Teste de citação: \cite{SistBib}

O Departamento de Engenharia Elétrica da Ufes possui diversos equipamentos, que podem ser utilizados durante as aulas práticas e também são utilizados pelos alunos em períodos fora do horário de aula nos laboratório para projetos a serem desenvolvidos como trabalho de conclusão do curso (TCC), trabalhos de diversas disciplinas, etc. Estes equipamentos costumam ter um alto custo, portanto, devemos ter muito cuidado com eles. O uso do equipamento é de extrema importância no desenvolvimento acadêmico dos alunos.
	

Durante as aulas práticas acompanhada por professores no laboratório, há empréstimos de Kits com cabos de osciloscópio e de fonte de alimentação e protoboard. Esse processo de empréstimo de Kits é controlado manualmente. Para adquirir o Kit, o aluno tem que entregar um documento com foto e só será devolvido na devolução do Kit. O funcionário fixa no documento do aluno uma ficha com número referente ao Kit emprestado. Esse processo acaba gerando filas no laboratório e confusão na organização dos documentos dos alunos. 

Ainda não há um controle para o empréstimos desse equimentos que serão utilizados fora dos laboratórios. Deve ser montando um sistema que auxilie e controle dos equipamentos nesse casos. Se for utilizado um sistema manual, haverá dificultadades para gerar e analisar o histórico de equipamentos. É presciso automatizar.

Além disso, o sistema guardião é um histórico de todos os empréstimos feitos para o acesso necessário. A automação do empréstimo pode ser feita com um sistema existente, como Aleph, Scobi, Pergamum, entre outros, mas esses sistemas são desenvolvidos para o armazenamento de bibliotecas, o que faz com que tenham funções que não são o caso do controle de empréstimos de equipamentos. . Além disso, os sistemas costumam ter um alto custo de implementação e alguns casos de hardware, o que aumenta ainda mais o custo (RODRIGUES e PRUDENCIO, 2009).

Trabalho atual com o desenvolvimento de um sistema de baixo custo para automatizar o processo de empréstimo associado ao dos Laboratórios da Engenharia Elétrica da Ufes.