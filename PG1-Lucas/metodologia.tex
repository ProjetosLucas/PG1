% Ajustar esse \vspace de acordo com o necessário 
\vspace{-42pt} 
O desenvolvimento do trabalho será dividido em etapas para modularizar, dividir em partes onde cada uma dessas partes será responsável pela realização de uma etapa do projeto. Esta forma de desenvolvimento teve como objetivo reduzir falhas no processo de desenvolvimento. Os estágios de desenvolvimento serão abordadas nas seções a seguir. 

\section[Estabelecer e Revisar as propriedades do sistema]{Estabelecer e Revisar as propriedades do sistema}
Alguns requisitos e características do sistema serão definidos, para que as atividades que o compõem o projeto sejam melhor direcionados. 

\section[Instalação do pacotes, programas e sistema operacional para o desenvolvimento do projeto]{Instalação do pacotes, programas e sistema operacional para o desenvolvimento do projeto}
Antes de iniciar o projeto, deve intalar o programas para o desenvolvimento do hardware (programação do microcontroladores, placas de circuito impresso, simulações dos circuitos) e para o desenvolvimento da interface WEB e banco de dados (o pacote XAMPP).

\section[Construção do Módulo leitor de código de barras]{Construção do Módulo leitor de código de barras}
Aqui será desenvolvido o circuito responsável pela leitura do código de barras do equipamento e da carteira do estudante que indentifica a sua matrícula. 

\section[Construção do Módulo GPS]{Construção do Módulo GPS}
Módulo que vai alimentar o banco de dados com a localização dos equipamentos. 

\section[Montando o banco de dados do sistema]{Montando o banco de dados do sistema}
Será modelado um banco de dados com todas as tabelas com colunas e relações nescessárias para atender o objetivo do projeto. 

\section[Desenvolvimento da interface web]{Desenvolvimento da interface web}
Com o banco de dados modelado, inicia o desenvolvimento da interface web do sistema. Para o desenvolvimento dessa interface irá ser utilizado principalmente framworks, que facilitam o desenvolviemnto.

\section[Integração do Hardware e Software]{Integração do Hardware e Software}

Será feito a integração do hardware e do software, com os dois funcionando em conjunto, fazendo os devidos concertos.

\section[Testes e validações]{Testes e validações}

Com o sistema em operação, realize as verificações e testes para validar sua operação de maneira estável. Com isso, o sistema foi validado e dado como pronto para ser utilizado no controle de empréstimos de equipamentos.

\section[Lições Aprendidas]{Lições Aprendidas}
Aqui irá registrar tudo que foi aprendido durante o desenvolvimento do projeto. Nesta parte será focado no relatório e na apresentação final.