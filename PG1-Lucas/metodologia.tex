% Ajustar esse \vspace de acordo com o necessário
\vspace{-42pt}
A partir da problemática apontada nas seções Introdução e Objetivos temos que o problema é detalhado nos parágrafos seguintes. 
O Laboratório mantém sob sua guarda, um número limitado de equipamentos patrimoniados, que disponibiliza para utilização pelos alunos de diversos curso visando auxiliar as aulas. 
Alguns materiais se desgastam pelo o próprio tempo de uso e outros pela falta de cuidado e zelo. Há duas categorias de objetos: LIVRE USO e USO RESTRITO. Na categoria de LIVRE USO organizamos os objetos que podem ser utilizados por todos os estudantes. Na categoria de USO RESTRITO organizamos os objetos que são mais dispendiosos e delicados. O AutomaTIK irá abranger somente os KITs (protoboard e cabos coaxiais para conexão do gerador de sinais e osciloscópio) que são usados durantes as aulas de laboratório. 
Com o intuito de normatizar a cessão desses bens e acreditando no uso consciente e responsável deste patrimônio pelos alunos, iremos fazer o AutomaTIK que segue o seguintes procedimento:
Primeiramente é necessário adicionar usuários ao banco de dados do sistema. Estas são pessoas que emprestam itens. Também tem que adicionar os itens.

\section[Cadastramento dos Alunos]{Cadastramento dos Alunos}  
Apenas alunos e professores cadastrados no programa de empréstimo estão habilitados a solicitar empréstimo de equipamentos, ficando responsável pelo bem durante o período de empréstimo. O cadastro deve ser feito antes de qualquer empréstimo que o aluno possa fazer, onde será inseridos os dados dos alunos como nome, matrícula e código do cartão RU obtido no sensor RFID. Para realizar o cadastramento, o aluno deverá procurar o responsável pela tarefa (funcionário ou monitor).
\section[Da verificação de disponibilidade]{Da verificação de disponibilidade}  
O sistema faz uma verificação automática se equipamento está disponível. Assim se estiver disponível e se o aluno estiver cadastrado, poderá validar o empréstimo. 
Ao tentar solicitar o empréstimo de um objeto e o mesmo já estiver alugado para outra pessoa, o novo solicitante será informado da condição do bem. Nesse ponto, o usuário pode fazer um pedido reserva de uso, assim quando o objeto estiver disponível o solicitante será notificado com um e-mail sobre a disponibilidade do objeto para seu futuro empréstimo. O novo solicitante tem o prazo de 1 (um) dia útil para que possa pegar o objeto de interesse. Caso o novo solicitante não venha pegar o objeto durante o prazo, sua reserva será cancelada.
\section[Da solicitação]{Da solicitação} 
Caso se os equipamentos estejam disponíveis e se o aluno ou professor estiver cadastrado, pode ser feita a solicitação do empréstimo na plataforma web com preenchimento de um formulário com os seguintes dados: a aula referida, números de matrícula ou código do cartão do RU (este último recebido pelo sensor RFID), código do equipamento requisitado e data e hora prevista de devolução. O número de matrícula vai ser dado pelo cartão do RU, através do sensor RFID. Também terá o campo de observações onde o usuário pode descrever possíveis anormalidades, como fios soltos, conexões errôneas, etc. dos equipamentos.
\section[Da retirada do equipamento]{Da retirada do equipamento} 
O próprio sistema irá informar se está autorizada a retirada do equipamento. Não autoriza-se caso o usuário está com algum equipamento pendente ou este veio com problemas.
\section[Do transporte]{Do transporte} 
A forma e o meio de retirada e de transporte para o local de utilização desses equipamentos são de responsabilidade do solicitante. 
\section[Da devolução do equipamento]{Da devolução do equipamento} 
O equipamento deverá ser devolvido à recepção do Laboratório dentro da data e hora agendadas.  Um profissional irá proceder à conferência física do material e irá confirmar no sistema se houve a devolução, adicionando observações de eventuais problemas ou danos ocasionados ao equipamento durante seu empréstimo. Se a devolução proceder sem problemas, o sistema irá mandar um recibo de devolução por e-mail do usuário.
Se o material for entregue danificado e possuir fila de espera, os solicitantes em aguardo irão ser informados do ocorrido. 
\section[Renovação]{Renovação} 
Caso deseje perdurar o prazo do objeto de empréstimo, o solicitante poderá fazer a Renovação. Basta ir no Laboratório e solicitar a renovação. Caso o objeto de empréstimo não possua fila de espera, a renovação poderá ser feita com sucesso. A quantidade máxima de renovação de um objeto é de 5 vezes.
