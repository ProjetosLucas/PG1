% Ajustar esse \vspace de acordo com o necessário
\vspace{-42pt}
Na parte de hardware, teremos que utilizar:
\begin{itemize}
   \item O servidor do LCEE que fará a armazenagem e processamento de dados; 
   \item Um leitor de RFID para registrar o login do usuário;
   \item Arduíno para viabilizar a comunicação do leitor RFID com o servidor;
   \item Tranca eletrônica para segurança dos equipamentos/KITs.
\end{itemize}
Já na parte de software podemos utilizar:
\begin{itemize}
   \item Um Framework PHP como o Laravel ou CakePHP para facilitar no desenvolvimento do sistema de login; 
   \item Banco de dados SQL (Structured Query Language).
\end{itemize}

Uma inspiração que temos para software é a plataforma web Lend-Itens. Abaixo podemos ver como ficará a plataforma Web:
Figura 1. Na plataforma Lend-Itens, os usuários podem acessar sua biblioteca para pesquisar um item e reservá-lo, bem como ver seu histórico e os empréstimos atuais.
Figura 2. Pode-se verificar quais são as pessoas que utilizam a plataforma.

Outras plataformas que podemos ter como base são Vaivem, apresentando a seguinte interface:

Figura 3. Interface de Vaivem.

Há outros softwares também como Software de Controle de UPJ e TotalLoc.
Também estamos utilizando o seguinte modelo de banco de dados para nosso projeto :

Figura 4. Configuração do Banco de Dados.
Também podemos ver quando o banco é acessado pelo seguintes esquemático:

Figura 5. Acesso do Banco de Dados.