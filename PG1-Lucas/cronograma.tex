% Ajustar esse \vspace de acordo com o necessário
\vspace{-42pt}

A seguir apresenta a lista contendo todas as atividades previstas.

\begin{enumerate}
	\item \textbf{Pesquisar e Revisar os Requisitos Para o Projeto:} Realizar pesquisas para encontrar o melhoras formas de desenvolver o projeto e também revisar os requisitos do projeto, fazendo sugestões de alterações que forem nescessárias.
	\item  \textbf{Discutir e Brainstorm:} Discutir com orientador as ideias encontradas em pesquisas e os requisitos do projeto;
	\item \textbf{Definir melhores meios, ferramentas e requisitos:} Aqui terá todos definidos todos os meios, ferramentas e requisitos, conclusão das tarefas 1 e 2;
	\item \textbf{Realizar a solicitação de materiais:} Aqui será feito a compra principalmente de componentes e microcontroladores nescessários para que seja montado os circuitos dos sistemas embarcados. Será adquirido também um servidor onde hospedará o banco de dados e a interface WEB.
	\item \textbf{Construir um banco de dados e Registrar as Lições Aprendidas:} Aqui será modelado um banco de dados com todas as tabelas, colunas e relações nescessárias para atender o objetivo do projeto. Terá que registrar as lições aprendidas para que seja abordado no relatório de TCC.
	\item \textbf{Construir a interface de interação direta com o usuário e Registrar as Lições Aprendidas:} Será desenvolvido uma interface que atende todos os requisitos do projeto, mas não importando com seu aspecto visual. Deve-se também registar as lições aprendidas.
	\item \textbf{Tornar apresentação do software mais amigável o possível e Registrar lições aprendidas:} Essa tarefa tem que fazer com que  o usuário consiga interagir com a interface de forma rápida, intuitiva e sem dificuldades suas funções. Aqui prescisa mecher no aspecto visual. 
	\item \textbf{Instalar e Configurar o Servidor para os responsáveis do laboratório terem acesso:}
	\item \textbf{Construir circuitos necessários para a utilização do RFID:}
	\item \textbf{Registrar Lições Aprendidas:}
	\item \textbf{Fazer a junção do hardware com software:}
	\item \textbf{Validar:}
	\item \textbf{Preparar para apresentação intermediária:}
	\item \textbf{Apresentação Intermediária:}
	\item \textbf{Fazer as modificações pedidas pelo professor:}
	\item \textbf{Realizar Últimos Testes:}
	\item \textbf{Preparar Apresentação Final:}
	\item \textbf{Apresentação Final:}
\end{enumerate}

Os quadros \ref{fig:quadro}, \ref{fig:quadro1}, \ref{fig:quadro2}, \ref{fig:quadro3} e \ref{fig:quadro4} mostra o tempo das atividades nos respectivos meses de julho, agosto, semtembro, outubro e novembro respectivamente.

\begin{quadro}[!h]
\centering
	\caption{Cronograma do Mês de Julho.}
	\begin{ganttchart}
	[y unit title=0.4cm,
	y unit chart=0.5cm,
	vgrid,hgrid,
	title height=1,
	bar/.style={draw,fill=cyan},
	bar incomplete/.append style={fill=yellow!50},
	bar height=0.7]{1}{31}
	 \gantttitle{Julho}{31}\\
	 \gantttitlelist{1,...,31}{1} \\
	 \ganttbar{Fase 1}{3}{7} \\
	 \ganttbar{Fase 2}{7}{12} \\
	 \ganttbar[progress=70]{Fase 3}{13}{18} \\
	 \ganttbar[progress=40]{Conclus\~ao}{20}{24} \\
	 % rela\c c\~oes
	 %\ganttlink{elem0}{elem1}
	 %\ganttlink{elem1}{elem2}
	 %\ganttlink{elem2}{elem3}
	\end{ganttchart}
	\source
	\label{fig:quadro}
\end{quadro}

\begin{quadro}[!h]
\centering
	\caption{Cronograma do Mês de Agosto.}
	\begin{ganttchart}
[y unit title=0.4cm,
y unit chart=0.5cm,
vgrid,hgrid,
title height=1,
bar/.style={draw,fill=cyan},
bar incomplete/.append style={fill=yellow!50},
bar height=0.7]{1}{31}
 \gantttitle{Agosto}{31}\\
 \gantttitlelist{1,...,31}{1} \\
 \ganttbar{Fase 1}{3}{7} \\
 \ganttbar{Fase 2}{7}{12} \\
 \ganttbar[progress=70]{Fase 3}{13}{18} \\
 \ganttbar[progress=40]{Conclus\~ao}{20}{24} \\
 % rela\c c\~oes
 %\ganttlink{elem0}{elem1}
 %\ganttlink{elem1}{elem2}
 %\ganttlink{elem2}{elem3}
\end{ganttchart}
	\source
	\label{fig:quadro1}
\end{quadro}

\begin{quadro}[!h]
\centering
	\caption{Cronograma do Mês de Setembro.}
	\begin{ganttchart}
[y unit title=0.4cm,
y unit chart=0.5cm,
vgrid,hgrid,
title height=1,
bar/.style={draw,fill=cyan},
bar incomplete/.append style={fill=yellow!50},
bar height=0.7]{1}{30}
 \gantttitle{Setembro}{30}\\
 \gantttitlelist{1,...,30}{1} \\
 \ganttbar{Fase 1}{3}{7} \\
 \ganttbar{Fase 2}{7}{12} \\
 \ganttbar[progress=70]{Fase 3}{13}{18} \\
 \ganttbar[progress=40]{Conclus\~ao}{20}{24} \\
 % rela\c c\~oes
 %\ganttlink{elem0}{elem1}
 %\ganttlink{elem1}{elem2}
 %\ganttlink{elem2}{elem3}
\end{ganttchart}
	\source
	\label{fig:quadro2}
\end{quadro}

\begin{quadro}[!h]
\centering
	\caption{Cronograma do Mês de Outubro.}
	\begin{ganttchart}
[y unit title=0.4cm,
y unit chart=0.5cm,
vgrid,hgrid,
title height=1,
bar/.style={draw,fill=cyan},
bar incomplete/.append style={fill=yellow!50},
bar height=0.7]{1}{31}
 \gantttitle{Outubro}{31}\\
 \gantttitlelist{1,...,31}{1} \\
 \ganttbar{Fase 1}{3}{7} \\
 \ganttbar{Fase 2}{7}{12} \\
 \ganttbar{Fase 3}{12}{22} \\
 \ganttbar{Fase 4}{22}{31} \\
 \ganttbar[progress=70]{Fase 3}{13}{18} \\
 \ganttbar[progress=40]{Conclus\~ao}{20}{24} \\
 % rela\c c\~oes
 %\ganttlink{elem0}{elem1}
 %\ganttlink{elem1}{elem2}
 %\ganttlink{elem2}{elem3}
\end{ganttchart}
	\source
	\label{fig:quadro3}
\end{quadro}

\begin{quadro}[!h]
\centering
	\caption{Cronograma do Mês de Novembro.}
	\begin{ganttchart}
[y unit title=0.4cm,
y unit chart=0.5cm,
vgrid,hgrid,
title height=1,
bar/.style={draw,fill=cyan},
bar incomplete/.append style={fill=yellow!50},
bar height=0.7]{1}{30}
 \gantttitle{Novembro}{30}\\
 \gantttitlelist{1,...,30}{1} \\
 \ganttbar{Fase 1}{3}{7} \\
 \ganttbar{Fase 2}{7}{12} \\
 \ganttbar[progress=70]{Fase 3}{13}{18} \\
 \ganttbar[progress=40]{Conclus\~ao}{20}{24} \\
 % rela\c c\~oes
 %\ganttlink{elem0}{elem1}
 %\ganttlink{elem1}{elem2}
 %\ganttlink{elem2}{elem3}
\end{ganttchart}
	\source
	\label{fig:quadro4}
\end{quadro}
